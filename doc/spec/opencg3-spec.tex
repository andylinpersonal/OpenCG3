\documentclass[10pt]{beamer}
\renewcommand\textbullet{\ensuremath{\bullet}}

\usepackage{sty/opencg3-spec}
\usepackage{sty/opencg3-spec-beamer}
\usepackage{xspace,bookmark,lmodern,mathtools,color,multido,ulem,upquote,tabularx}
\usepackage[T1]{fontenc}
\usefonttheme[onlymath]{serif}

\newcommand{\Version}[0]{Version 0.2.6}
\newcommand{\PrjName}{OpenCG\texorpdfstring{\textsuperscript{3}}}
\newcommand{\PrjNameFull}{Open Command-oriented\\Geometric Graphics Generator}
\newcommand{\PrjSpec}{\PrjName{} Spec \Version}

\title[\PrjSpec]{\PrjNameFull}
\subtitle{\small \PrjSpec}


\author[KVD \and ADL]{Dong Nai-Jia \inst{1} \and Lin Yong-Hsiang \inst{2}}
\institute{	\inst{1} National Chiao Tung University\\Department of Computer Science \and
			\inst{2} National Taiwan University\\Department of Agricultural Chemistry}
\date[\today]{\today}


\begin{document}


\begin{frame}
	\titlepage
\end{frame}


\section{Overview}

%\subsection{Illustration}

%\begin{frame} \frametitle{Perspective Projection}

%	\begin{figure}[t]
%		\includegraphics[width=.5\columnwidth]{fig/perspective-projection.png}
%		\caption{Projection in Euclidean $\mathbb{R}^3$ Space}
%	\end{figure}

%\end{frame}

\subsection{Definitions}

\begin{frame}[t] \frametitle{Command Tokens}

	\begin{block}{Regular Expressions}
		$\MSN \coloneqq \{\ \alpha \mid \alpha \in \texttt{[0-9]+} \ \}$ \\ [.24em]
		$\MSR \coloneqq \{\ \alpha \mid \alpha \in \texttt{[+\textbackslash-]?([0-9]*[.])?[0-9]+}\ \}$ \hfill
		$\Rightarrow \MSR \supset \MSN \ $ \\ [.24em]
		$\MSS \coloneqq \{\ \alpha \mid \alpha \in \texttt{\SingleQuote(.*?)\SingleQuote|[.0-9A-Za-z+\textbackslash-]+} \ \}$ \hfill
		$\Rightarrow \MSS \supset \MSR \ $ \\ [.24em]
		$\MSW \coloneqq \{\ \alpha \mid \alpha \in \texttt{[ \textbackslash{}t]} \}$ \hfill whitespace
	\end{block}

	\begin{block}{Descriptions}
		\begin{itemize}
			\item The matching mechanism abides by the maximal munch rule.
			\item Each command is whitespace-insensitive except being quoted by a pair of single quotation marks (\SingleQuote).
		\end{itemize}
	\end{block}

\end{frame}

\begin{frame}[t] \frametitle{Command Grammars}

	\begin{block}{Context-Free Expansions}
		$\left.\CMD \EXP \ARG \CMD \SEP \,\text{;}\, \SEP \,\texttt{EOL} \right.$ \\ [.24em]
		$\left.\ARG \EXP \TUP(\ARG) \SEP \VCT(\ARG) \SEP \SET(\ARG) \SEP
		 \LST(\ARG) \SEP \LST(\ARG, \ARG, \cdots, \ARG) \SEP \MSN \SEP \MSR \SEP \MSS \right.$ \\ [.27em]
		$\left.\text{%
		\begin{tabular}{@{}l}%
			$\TUP(\Pi) \EQV \Tup{\Pi}{n} \EXP \texttt{\TupBrkL}     \ \, \Sigma(\Pi, n) \ \, \texttt{\TupBrkR}$ \\ [.24em]
			$\VCT(\Pi) \EQV \Vct{\Pi}{n} \EXP \texttt{\VctBrkTextL} \ \, \Sigma(\Pi, n) \ \, \texttt{\VctBrkTextR}$ \\ [.24em]
			$\SET(\Pi) \EQV \Set{\Pi}{n} \EXP \texttt{\SetBrkL}     \ \, \Sigma(\Pi, n) \ \, \texttt{\SetBrkR}$
		\end{tabular}} \right\|\kern.128em$%
		\begin{tabular}{@{}l}%
			$\ \Sigma(\Pi, n) \EXP \overbrace{\Pi \ \cdots \ \Pi}^{n \ \text{items}} \quad\,\ \text{(identical)}$ \\ [.43em]
			$\ \LST(\Pi) \EQV \Lst{\Pi}{n} \EXP \texttt{\LstBrkL} \ \, \Sigma(\Pi, n) \ \, \texttt{\LstBrkR}$
		\end{tabular} \\ [.24em]
		$\left.\LST(\Pi_1,\Pi_2,\cdots,\Pi_{n-1},\Pi_n) \EQV \LstFull{\Pi_1\,\Pi_2\cdots\Pi_{n-1}\,\Pi_n} \EXP
		 \texttt{\LstBrkL} \ \, \Pi_1\cdots\Pi_n \ \, \texttt{\LstBrkR}\right.$
	\end{block}

	\begin{block}{Descriptions}
		\begin{itemize}
			\item Each command starts from $\CMD$ and ends with a \texttt{;} or an \texttt{EOL}.
			\item Non-terminal symbol expansions are prior than function expansions except that symbols are used for describing arguments of a command.
		\end{itemize}
	\end{block}

\end{frame}

\begin{frame}[t] \frametitle{Command Parsing}

	\begin{block}{Escape Sequence}
		\begin{itemize}
			\item \texttt{\textbackslash x} is an escape sequence.
			\item If \texttt{x} is \texttt{\textbackslash}, then it is treated as a single backslash.
			\item If \texttt{x} is \texttt{EOL} which may vary from platforms, then the sequence is omitted.
			\item Otherwise, the sequence is ignored and triggers a warning by default.
		\end{itemize}
	\end{block}

	\begin{block}{Error Handling}
		\begin{itemize}
			\item Physical lines are separated by an \texttt{EOL}.
			\item Logical lines are separated by either a semicolon or an unescaped \texttt{EOL}.
			\item If the command cannot be parsed by the grammar, then all the characters on the same logical line will be discarded.
		\end{itemize}
	\end{block}

\end{frame}

\begin{frame}[t] \frametitle{Class and Object System}

	\begin{block}{Classes}
		\begin{itemize}
			\item Classes are split into two categories, top and bottom.
			\item Top classes consist of window, camera, point, line, surface, etc.
			\item Bottom classes consist of attrib(ute) and group.
		\end{itemize}
	\end{block}

	\begin{block}{Objects}
		\begin{itemize}
			\item An object is derived from a class aforementioned.
			\item An object has an unique name throughout its class category derivations.
		\end{itemize}
	\end{block}

	\begin{block}{Relations}
		\begin{itemize}
			\item Objects derived from the same class category cannot form a relation.
			\item Relations are bidirectional and can be created or deleted via commands.
		\end{itemize}
	\end{block}

\end{frame}


\section{Commands}

\subsection{Window}

\begin{frame}[t] \frametitle{Create a Window}

	\begin{block}{Command} \newcolumntype{R}{>{\raggedleft\arraybackslash}X}
		\begin{tabularx}{\textwidth}{@{}l@{}l@{}l@{}l@{}R}
			\InstrName{create window} &
				\ParamMust{\StrName{label}} &
				\ParamOptl{\TupName{\MSR}{3}{coord}} &
				\ParamOptl{\TupName{\Vct{\MSR}{3}}{3}{dirct}} & \InstrItem
		\end{tabularx}
	\end{block}

	\begin{block}{Parametres} \begin{itemize}
		\ParamItem{label} the object name of the class window
		\ParamItem{coord} the coordinate $(c_x, c_y, c_z)$ of the centre of the window.
		\ParamItem{dirct} the window width $\vec{v_w}$, height $\vec{v_h}$, and the camera view $\vec{v_c}$.
	\end{itemize} \end{block}

	\begin{block}{Examples}
		\CommandEx{create window main \TupText{0 0 1} \TupText{\VctText{1 0 0} \VctText{0 1 0} \VctText{0 0 1}}}
	\end{block}

\end{frame}

\begin{frame}[t] \frametitle{Delete a Window}

	\begin{block}{Command} \newcolumntype{R}{>{\raggedleft\arraybackslash}X}
		\begin{tabularx}{\textwidth}{@{}l@{}l@{}R}
			\InstrName{delete window} &
			  	\ParamOptl{\StrName{message}} & \InstrItem
		\end{tabularx}
	\end{block}

	\begin{block}{Parametres} \begin{itemize}
		\ParamItem{message} the text string printed right after exiting
	\end{itemize} \end{block}

	\begin{block}{Examples}
		\CommandEx{delete window}
		\CommandEx{delete window \SingleQuote Have a nice day.\SingleQuote}
	\end{block}

\end{frame}

\subsection{Points}

\begin{frame}[t] \frametitle{Create Points}

	\begin{block}{Command} \newcolumntype{R}{>{\raggedleft\arraybackslash}X}
		\begin{tabularx}{\textwidth}{@{}l@{}l@{}l@{}R}
			\InstrName{create point } &
			  	\ParamMust{\SetOptl{\StrName{label}}{}} &
			  	\ParamMust{\TupName{\MSR}{3}{coord}} & \InstrItem \\
			\InstrName{create point } &
				\ParamMust{\Tup{\StrName{label}}{\,\geqslant n}} &
				\ParamMust{\Tup{\TupName{\MSR}{3}{coord}}{n}} & \InstrItem
		\end{tabularx}
	\end{block}

	\begin{block}{Parametres} \begin{itemize}
		\ParamItem{label} the object name of the class point
		\ParamItem{coord} the coordinate $(p_x, p_y, p_z)$ of the point
	\end{itemize} \end{block}

	\begin{block}{Examples}
		\CommandEx{create point \SingleQuote origin\SingleQuote \ \ \ \TupText{0 0 0}}
		\CommandEx{create point \SetText{X-1 X-2} \ \TupText{1 0 0}}
		\CommandEx{create point \TupText{Y-1 Z-1} \TupText{\TupText{0 1 0}\TupText{0 0 1}}}
	\end{block}

\end{frame}

\begin{frame}[t] \frametitle{Delete Points}

	\begin{block}{Command} \newcolumntype{R}{>{\raggedleft\arraybackslash}X}
		\begin{tabularx}{\textwidth}{@{}l@{}l@{}R}
			\InstrName{delete point } &
				\ParamMust{\SetOptl{\StrName{label}}{}} & \InstrItem
		\end{tabularx}
	\end{block}

	\begin{block}{Parametres} \begin{itemize}
		\ParamItem{label} the object name of the class point
	\end{itemize} \end{block}

	\begin{block}{Examples}
		\CommandEx{delete point \ origin}
		\CommandEx{delete point \SetText{origin \SingleQuote random-point\SingleQuote}}
	\end{block}

\end{frame}

\subsection{Attribute}

\begin{frame}[t] \frametitle{Create Attributes}

	\begin{block}{Command} \newcolumntype{R}{>{\raggedleft\arraybackslash}X}
		\begin{tabularx}{\textwidth}{@{}l@{}l@{}l@{}R}
			\InstrName{create attrib} &
				\ParamMust{\SetOptl{\StrName{desc}}{}} &
				\ParamMust{\LstOptl{\LstFull{\StrName{t-class} \ \StrName{key} \ \ArgName{value}}}{}} & \InstrItem \\
			\InstrName{create attrib} &
				\ParamMust{\Tup{\StrName{desc}}{}} &
				\ParamMust{\Lst{\LstFull{\StrName{t-class} \ \StrName{key} \ \ArgName{value}}}{}} & \InstrItem
		\end{tabularx}
	\end{block}

	\begin{block}{Parametres} \begin{itemize}
		\ParamItem{desc} the object name of the class attrib
		\ParamItem{t-class} the name of one of the top classes
		\ParamItem{key} the property of the object of class \Param{t-class}
		\ParamItem{value} the appropriate value of the property \Param{key}
	\end{itemize} \end{block}

	\begin{block}{Examples}
		\CommandEx{create attrib \TupText{magenta dashed-and-traslucent-green} \EOLText
		                         \LstText{\LstText{point fill-hsv \TupText{300 1.0 1.0}} \EOLText
		                       \ \LstText{line style dashed} \LstText{line fill-rgba \LstText{\TupText{0 255 0} .5}}}}
	\end{block}

\end{frame}

\begin{frame}[t] \frametitle{Attach Attributes}

	\begin{block}{Command} \newcolumntype{R}{>{\raggedleft\arraybackslash}X}
		\begin{tabularx}{\textwidth}{@{}l@{}l@{}l@{}R}
			\InstrName{attach attrib} &
				\ParamMust{\TupOptl{\StrName{desc}}{}} &
				\ParamMust{\SetOptl{\StrName{label}}{}} & \InstrItem \\
			\InstrName{attach attrib} &
				\ParamMust{\Tup{\StrName{desc}}{}} &
				\ParamMust{\Tup{\StrName{label}}{}} & \InstrItem
		\end{tabularx}
	\end{block}

	\begin{block}{Parametres} \begin{itemize}
		\ParamItem{desc} the name of the object of the class attrib
		\ParamItem{label} the name of the object derived from the top classes
	\end{itemize} \end{block}

	\begin{block}{Examples}
		\CommandEx{attach attrib \ red \quad \quad \quad \ \ point-0}
		\CommandEx{attach attrib \TupText{red large} \ point-1}
		\CommandEx{attach attrib \ blue \quad \quad \quad \SetText{point-2 rect-0}}
		\CommandEx{attach attrib \TupText{5px black} \SetText{point-3 circ-0}}
		\CommandEx{attach attrib \TupText{red thick} \TupText{point-4 line-0 trianle-0}}
	\end{block}

\end{frame}

\subsection{Miscellaneous}

\begin{frame}[t] \frametitle{Assign an Operation Name}

	\begin{block}{Command} \newcolumntype{R}{>{\raggedleft\arraybackslash}X}
		\begin{tabularx}{\textwidth}{@{}l@{}l@{}l@{}l@{}R}
			\InstrName{assign opname} &
				\ParamMust{\StrName{action}} &
				\ParamMust{\StrName{class}} &
				\ParamOptl{\MSNName{repeat}\ \ValueDefn{=\infty}} & \InstrItem
		\end{tabularx}
	\end{block}

	\begin{block}{Parametres} \begin{itemize}
		\ParamItem{action} the name of the action
		\ParamItem{class} the name of one of the classes
		\ParamItem{repeat} the amount of the commands emitting operation names
	\end{itemize} \end{block}

	\begin{block}{Examples}
		\CommandEx{assign instr create point 2}
		\CommandEx{x-axis \TupText{1 0 0}; y-axis \TupText{0 1 0}}
		\CommandEx{// Back To Normal}
	\end{block}

\end{frame}


\end{document}

